%%%%%%%%%%%%%%%%%%%%%%%%%%%%%%%%%%%%%%%%%%%%%%%%%%%%%%%%%%%%%%%%%%%%%%
\subsection{\label{sec:PrivSep}Privilege Separation}
%%%%%%%%%%%%%%%%%%%%%%%%%%%%%%%%%%%%%%%%%%%%%%%%%%%%%%%%%%%%%%%%%%%%%%
\index{privilege separation}
\index{PrivSep (privilege separation)}
\index{HTCondor daemon!condor\_procd@\Condor{procd}}
\index{daemon!condor\_procd@\Condor{procd}}
\index{condor\_procd daemon}

Section~\ref{sec:uids} discusses why, under most circumstances, it is
beneficial to run the HTCondor daemons as \Login{root}. In situations
where multiple users are involved or where HTCondor is responsible for
enforcing a machine owner's policy, running as \Login{root} is the 
\emph{only} way for HTCondor to do its job correctly and securely.

Unfortunately, this requirement of running HTCondor as \Login{root} is
at odds with a well-established goal of security-conscious
administrators: keeping the amount of software that runs with
superuser privileges to a minimum. HTCondor's nature as a
large distributed system that routinely communicates with potentially
untrusted components over the network further aggravates this goal.

The privilege separation (PrivSep) effort in HTCondor aims to minimize
the amount of code that needs \Login{root}-level access, while still
giving HTCondor the tools it needs to work properly. Note that PrivSep
is currently only available for execute side functionality, and is not
implemented on Windows.

In the PrivSep model, all logic in HTCondor that requires superuser
privilege is contained in a small component called 
the PrivSep Kernel.
The HTCondor daemons execute as an unprivileged account.
They explicitly request action from the PrivSep Kernel whenever
\Login{root}-level operations are needed.

The PrivSep model then prevents the following attack scenario.
In the attack scenario, an attacker has found an exploit in the
\Condor{startd} that allows for execution of arbitrary code on that
daemon's behalf. 
This gives the attacker \Login{root} access and therefore
control over any machine
on which the \Condor{startd} is running as \Login{root}
and the exploit can be exercised.
Under the PrivSep model,
the \Condor{startd} no longer runs as \Login{root}.
This prevents the attacker from taking arbitrary action as \Login{root}. 
Further, limits on requested actions from the PrivSep Kernel
contain and restrict the attacker's sphere of influence.

The following section describes the configuration necessary
to enable PrivSep for an execute-side HTCondor installation.
After this is a detailed description of
the services that the PrivSep Kernel provides to HTCondor, and how it
limits the allowed \Login{root}-level actions.

%%%%%%%%%%%%%%%%%%%%%%%%%%%%%%%%%%%%%%%%%%%%%%%%%%%%%%%%%%%%%%%%%%%%%%
\subsubsection{PrivSep Configuration}
%%%%%%%%%%%%%%%%%%%%%%%%%%%%%%%%%%%%%%%%%%%%%%%%%%%%%%%%%%%%%%%%%%%%%%

The PrivSep Kernel is implemented as two programs:
the \Condor{root\_switchboard} and the \Condor{procd}.
Both are contained in
the \File{sbin} directory of the HTCondor distribution. When HTCondor is
running in PrivSep mode, these are to be the only two HTCondor daemons that run
with \Login{root} privilege.

Each of these binaries must be accessible on the file system via a
\Term{trusted path}. A trusted path ensures that no user (other than
\Login{root}) can alter the binary or path to the binary referred to.
To ensure that the
paths to these binaries are trusted, use only
\Login{root}-owned directories, and set the permissions on
these directories to deny write access to all but \Login{root}. 
The binaries themselves must also be owned by \Login{root} 
and not writable by any other.
The \Condor{root\_switchboard} program additionally
is installed with the setuid bit set. The following command
properly sets the permissions on the \Condor{root\_switchboard}
binary:
\begin{verbatim}
chmod 4755 /opt/condor/release/sbin/condor_root_switchboard
\end{verbatim}

The PrivSep Kernel has its own configuration file.
This file  must be \File{/etc/condor/privsep\_config}.
The format of this file is different than 
an HTCondor configuration file.
It consists
of lines with ``\Code{key = value}'' pairs. 
Lines with only white space
or lines with ``\Code{\#}'' as the first non-white space character are ignored.

In the PrivSep Kernel configuration file,
some configuration settings are interpreted as single values, 
while others are interpreted as lists. 
To populate a list with multiple values,
use multiple lines with the same key.
For example, the following configures the \Code{valid-dirs}
setting as a list with two entries:
\begin{verbatim}
valid-dirs = /opt/condor/execute_1
valid-dirs = /opt/condor/execute_2
\end{verbatim}
It is an error to have multiple lines with the same key for a setting
that is not interpreted as a list.

Some PrivSep Kernel configuration file settings require a list of UIDs or GIDs,
and these allow for a more specialized syntax.
User and group IDs can be specified either numerically or textually.
Multiple list entries may be given on a single
line using the \Code{:} (colon) character as a delimiter.
In addition, list entries
may specify a range of IDs using a \Code{-} (dash)
character to separate the minimum and maximum IDs included.
The \Code{*} (asterisk) character on the right-hand side of such a
range indicates that the range extends to the maximum possible ID. 
The following example builds a complex list of IDs:
\begin{verbatim}
valid-target-uids = nobody : nfsuser1 : nfsuser2
valid-target-uids = condor_run_1 - condor_run_8
valid-target-uids = 800 - *
\end{verbatim}

If \Code{condor\_run\_1} maps to UID 701, and
\Code{condor\_run\_8} maps to UID 708, 
then this range specifies the 8 UIDs of 701 through 708 (inclusive).

The following settings are required to configure the PrivSep Kernel:
\begin{itemize}

\item \Code{valid-caller-uids} and \Code{valid-caller-gids}. These lists
specify users and groups that will be allowed to request action from
the PrivSep Kernel. The list typically will contain the
UID and primary GID that the HTCondor daemons will run as.

\item \Code{valid-target-uids} and \Code{valid-target-gids}. These lists
specify the users and groups that HTCondor will be allowed to act on
behalf of. The list will need to include IDs of all users and groups
that HTCondor jobs may use on the given execute machine.

\item \Code{valid-dirs}. This list specifies directories that
HTCondor will be allowed to manage for the use of temporary job
files. Normally, this will only need to include the value of HTCondor's
\MacroU{EXECUTE} directory. Any entry in this list must be a trusted
path. This means that all components of the path must be directories
that are \Login{root}-owned and only writable by \Login{root}. For
many sites, this may require a change in ownership and permissions
to the \MacroU{LOCAL\_DIR} and \MacroU{EXECUTE} directories. Note also
that the PrivSep Kernel does not have access to HTCondor's configuration
variables, and therefore may not refer to them in this file.

\item \Code{procd-executable}. A (trusted) full path to the
\Condor{procd} executable.
Note that the PrivSep Kernel does not
have access to HTCondor's configuration variables,
and therefore may not refer to them in this file.

\end{itemize}

Here is an example of a full \File{privsep\_config} file.
This file gives the \Login{condor} account access to the PrivSep
Kernel. HTCondor's use of this execute machine will be restricted to a
set of eight dedicated accounts, along with the \Login{users}
group. HTCondor's \MacroUNI{EXECUTE} directory and the \Condor{procd}
executable are also specified, as required.
\begin{verbatim}
valid-caller-uids = condor
valid-caller-gids = condor
valid-target-uids = condor_run_1 - condor_run_8
valid-target-gids = users : condor_run_1 - condor_run_8
valid-dirs = /opt/condor/local/execute
procd-executable = /opt/condor/release/sbin/condor_procd
\end{verbatim}

Once the PrivSep Kernel is properly installed and configured, HTCondor's
configuration must be updated to specify that PrivSep should be
used. The HTCondor configuration variable \Macro{PRIVSEP\_ENABLED} 
is a boolean flag serving this purpose.
In addition, HTCondor must be told where the
\Condor{root\_switchboard} binary is located using the
\Macro{PRIVSEP\_SWITCHBOARD} setting. The following example
illustrates:
\begin{verbatim}
PRIVSEP_ENABLED = True
PRIVSEP_SWITCHBOARD = $(SBIN)/condor_root_switchboard
\end{verbatim}

Finally, note that while the \Condor{procd} is in general an optional
component of HTCondor, it is required when PrivSep is in use. If
\MacroNI{PRIVSEP\_ENABLED} is \Expr{True}, the \Condor{procd} will be used
regardless of the \Macro{USE\_PROCD} setting.
Details on these HTCondor configuration variables are in
section~\ref{sec:Config-PrivSep} for PrivSep variables
and 
section~\ref{sec:Procd-Config-File-Entries} for \Condor{procd} variables.

%%%%%%%%%%%%%%%%%%%%%%%%%%%%%%%%%%%%%%%%%%%%%%%%%%%%%%%%%%%%%%%%%%%%%%
\subsubsection{PrivSep Kernel Interface}
%%%%%%%%%%%%%%%%%%%%%%%%%%%%%%%%%%%%%%%%%%%%%%%%%%%%%%%%%%%%%%%%%%%%%%

This section describes the \Login{root}-enabled operations that the
PrivSep Kernel makes available to HTCondor. The PrivSep Kernel's
interface is designed to provide only operations needed by HTCondor in
order to function properly. Each operation is further restricted based
on the PrivSep Kernel's configuration settings.

The following list describes each action that can be performed via the
PrivSep Kernel, along with the limitations enforced on how it may be
used. The terms \emph{valid target users}, \emph{valid target groups},
and \emph{valid directories} refer respectively to the settings for
\Code{valid-target-uids}, \Code{valid-target-gids}, and
\Code{valid-dirs} from the PrivSep Kernel's configuration.

\begin{itemize}

\item \emph{Make a directory as a user.} This operation creates
an empty directory, owned by a user. The user must be a valid target
user, and the new directory's parent must be a valid directory.

\item \emph{Change ownership of a directory tree.} This operation
involves recursively changing ownership of all files and
subdirectories contained in a given directory. The directory's parent
must be a valid directory, and the new owner must either be a valid
target user or the user invoking the PrivSep Kernel.

\item \emph{Remove a directory tree.} This operation deletes a given directory,
including everything contained within. The directory's parent must be
a valid directory.

\item \emph{Execute a program as a user.} HTCondor can invoke the
PrivSep kernel to execute a program as a valid target user. The user's
primary group and any supplemental groups that it is a member of must
all be valid target groups. This operation may also include opening
files for standard input, output, and error before executing the
program.

\end{itemize}

After launching a program as a valid target user, the PrivSep Kernel
allows HTCondor limited control over its execution. The following
operations are supported on a program executed via the PrivSep Kernel:

\begin{itemize}

\item \emph{Get resource usage information.} This allows HTCondor to
gather usage statistics such as CPU time and memory image size. This
applies to the program's initial process and any of its descendants.

\item \emph{Signal the program.} HTCondor may ask that signals be sent
to the program's initial process as a notification mechanism.

\item \emph{Suspend and resume the program.} These operations send
\Code{SIGSTOP} or \Code{SIGCONT} signals to all processes that make up
the program.

\item \emph{Kill the process and all descendants.} HTCondor is allowed
to terminate the execution of the program or any processes
left behind when the program completes.

\end{itemize}

By sufficiently constraining the valid target
accounts and valid directories to which the PrivSep Kernel allows
access, the ability of a compromised HTCondor daemon to do damage can
be considerably reduced.

%%%%%%%%%%%%%%%%%%%%%%%%%%%%%%%%%%%%%%%%%%%%%%%%%%%%%%%%%%%%%%%%%%%%%%
\subsection{\label{sec:glexec}Support for \Prog{glexec}}
%%%%%%%%%%%%%%%%%%%%%%%%%%%%%%%%%%%%%%%%%%%%%%%%%%%%%%%%%%%%%%%%%%%%%%

\Prog{glexec} is a tool that provides a sudo-like capability in a grid
environment. 
\Prog{glexec} takes an X.509 proxy and a command to run as
inputs,
and maps the proxy to a local identity (that is, a Unix UID),
which it then uses to execute the command.
Like the \Condor{root\_switchboard} command,
which provides similar functionality for HTCondor's PrivSep mode
(see section \ref{sec:PrivSep}), \Prog{glexec} must be installed as a root-owned
setuid program. See \URL{http://www.nikhef.nl/grid/lcaslcmaps/glexec/}
for more information about \Prog{glexec}.

HTCondor can interoperate with \Prog{glexec}, using it in a similar way
to how the \Condor{root\_switchboard} is used when running HTCondor in
PrivSep mode. The \Condor{starter} uses \Prog{glexec} when launching
a job, in order to give the job a separate UID from that of the HTCondor daemons.
\Prog{glexec} is also used when performing maintenance actions such as
cleaning up a job's files and processes, which cannot be done well
directly under the HTCondor daemons' UID due to permissions.
A consequence
of this type of integration with \Prog{glexec} is that the execution
of a single HTCondor job results in several \Prog{glexec} invocations, and
each must map the proxy to the same UID.
It is thus important to ensure
that \Prog{glexec} is configured to provide this guarantee.

Configuration for \Prog{glexec} support is straightforward. 
The boolean configuration variable \Macro{GLEXEC\_JOB} 
must be set \Expr{True} on execute machines where \Prog{glexec} is to be used.
HTCondor also must be given the full path
to the \Prog{glexec} binary using the \Macro{GLEXEC} configuration variable.
Note that HTCondor must be started as a non-root user when \Prog{glexec} is used.
This is because when HTCondor runs as root,
it can perform actions as other UIDs arbitrarily,
and \Prog{glexec}'s services are not needed.
Finally, for a job to execute
properly in the mode utilizing \Prog{glexec},
the job must be submitted with a proxy
specified via the \SubmitCmd{x509userproxy} command in
its submit description file,
since a proxy is needed as input to \Prog{glexec}.

Earlier versions of HTCondor employed a different form of \Prog{glexec}
support, where the \Condor{starter} daemon ran under the same UID as the job.
This feature was enabled using 
the \Macro{GLEXEC\_STARTER} configuration variable.
This configuration variable is no longer used,
and it is replaced by the \MacroNI{GLEXEC\_JOB} configuration variable,
to enable usage of \Prog{glexec}.
