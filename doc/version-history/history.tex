%%%%%%%%%%%%%%%%%%%%%%%%%%%%%%%%%%%%%%%%%%%%%%%%%%%%%%%%%%%%%%%%%%%%%%
\section{\label{sec:History-Intro}Introduction to HTCondor Versions}
%%%%%%%%%%%%%%%%%%%%%%%%%%%%%%%%%%%%%%%%%%%%%%%%%%%%%%%%%%%%%%%%%%%%%%

This chapter provides descriptions of what features have been added or
bugs fixed for each version of HTCondor.
The first section describes the HTCondor version numbering scheme, what
the numbers mean, and what the different \Term{release series} are.
The rest of the sections each describe a specific release series, and
all the HTCondor versions found in that series.

%%%%%%%%%%%%%%%%%%%%%%%%%%%%%%%%%%%%%%%%%%%%%%%%%%%%%%%%%%%%%%%%%%%%%%
\subsection{\label{sec:Version-Number-Scheme}
HTCondor Version Number Scheme}
%%%%%%%%%%%%%%%%%%%%%%%%%%%%%%%%%%%%%%%%%%%%%%%%%%%%%%%%%%%%%%%%%%%%%%

Starting with version 6.0.1, HTCondor adopted a new, hopefully easy to
understand version numbering scheme.
It reflects the fact that HTCondor is both a production system and a
research project.
The numbering scheme was primarily taken from the Linux kernel's
version numbering, so if you are familiar with that, it should seem
quite natural.

There will usually be two HTCondor versions available at any given time,
the \Term{stable} version, and the \Term{development} version.
Gone are the days of ``patch level 3'', ``beta2'', or any other random
words in the version string.
All versions of HTCondor now have exactly three numbers, separated by
``.''   

\begin{itemize}

\item The first number represents the major version number, and will
change very infrequently.

\item \emph{The thing that determines whether a version of HTCondor is
\Term{stable} or \Term{development} is the second digit.
Even numbers represent stable versions, while odd numbers represent
development versions.}

\item The final digit represents the minor version number, which
defines a particular version in a given release series.

\end{itemize}


%%%%%%%%%%%%%%%%%%%%%%%%%%%%%%%%%%%%%%%%%%%%%%%%%%%%%%%%%%%%%%%%%%%%%%
\subsection{\label{sec:Stable-Series}The Stable Release Series}
%%%%%%%%%%%%%%%%%%%%%%%%%%%%%%%%%%%%%%%%%%%%%%%%%%%%%%%%%%%%%%%%%%%%%%

People expecting the stable, production HTCondor system should download
the stable version, denoted with an even number in the second digit of
the version string.
Most people are encouraged to use this version.  
We will only offer our paid support for versions of HTCondor from the
stable release series.

\emph{On the stable series, new minor version releases will only
be made for bug fixes and to support new platforms.}
No new features will be added to the stable series.
People are encouraged to install new stable versions of HTCondor when
they appear, since they probably fix bugs you care about.
Hopefully, there will not be many minor version releases for any given
stable series.


%%%%%%%%%%%%%%%%%%%%%%%%%%%%%%%%%%%%%%%%%%%%%%%%%%%%%%%%%%%%%%%%%%%%%%
\subsection{\label{sec:Developement-Series}
The Development Release Series}
%%%%%%%%%%%%%%%%%%%%%%%%%%%%%%%%%%%%%%%%%%%%%%%%%%%%%%%%%%%%%%%%%%%%%%

Only people who are interested in the latest research, new features
that haven't been fully tested, etc, should download the development
version, denoted with an odd number in the second digit of the version
string.  
We will make a best effort to ensure that the development series will
work, but we make no guarantees.

On the development series, new minor version releases will probably
happen frequently.
People should not feel compelled to install new minor versions unless
they know they want features or bug fixes from the newer development
version.

\emph{Most sites will probably never want to install a development
version of HTCondor for any reason.}
Only if you know what you are doing (and like pain), or were
explicitly instructed to do so by someone on the HTCondor Team, should
you install a development version at your site.

After the feature set of the development series is satisfactory to the
HTCondor Team, we will put a code freeze in place, and from that point
forward, only bug fixes will be made to that development series.
When we have fully tested this version, we will release a new stable
series, resetting the minor version number, and start work on a new
development release from there.

%%%%%%%%%%%%%%%%%%%%%%%%%%%%%%%%%%%%%%%%%%%%%%%%%%%%%%%%%%%%%%%%%%%%%%
% The rest of this file just inputs other files which contain sections
% describing each release series in detail.
%%%%%%%%%%%%%%%%%%%%%%%%%%%%%%%%%%%%%%%%%%%%%%%%%%%%%%%%%%%%%%%%%%%%%%

% upgrade instructions are in the Pool Management section
%%%%%%%%%%%%%%%%%%%%%%%%%%%%%%%%%%%%%%%%%%%%%%%%%%%%%%%%%%%%%%%%%%%%%%
\section{\label{sec:gotchas}Upgrading from the 7.6 series to the 7.8 series of HTCondor}
%%%%%%%%%%%%%%%%%%%%%%%%%%%%%%%%%%%%%%%%%%%%%%%%%%%%%%%%%%%%%%%%%%%%%%

\index{upgrading!items to be aware of}
While upgrading from the 7.6 series of HTCondor to the 7.8 series will bring many
new features and improvements introduced in the 7.7 series of HTCondor, it will
also introduce changes that administrators of sites running from an older
HTCondor version should be aware of when planning an upgrade.  Here is a list of
items that administrators should be aware of.

\begin{itemize}

\item In the grid universe, the Amazon grid-type is gone and has been replaced
	with the EC2 grid-type.  Also, support for grid-type gt4 (Web Services
	GRAM) has been removed.

\item Default job submit options related to file transfers have changed. 
Across all platforms, defaults are now
\begin{verbatim}
  should_transfer_files = IF_NEEDED
  when_to_transfer_output = ON_EXIT
\end{verbatim}
See section~\ref{sec:file-transfer-if-when} for details.

\item  On Linux and Mac OS, common utility code is now contained in a set of
shared libraries. In the Linux native packages, most of these libraries
are placed under \File{/usr/lib[64]/condor} and the RUNPATH attribute is set in
the binaries to search there for them.
In the tarball packages, these libraries are placed under \File{lib} and
\File{lib/condor}, and the RUNPATH attribute is set in the binaries to search
for them under the relative paths \File{../lib} and \File{../lib/condor}.
This means that if you move or copy an HTCondor binary from a tarball
package to a different location, you must do one of the following:
\begin{itemize}
	\item Move or copy the corresponding \File{lib/} directory with it, or
  \item Make a symlink in the new location pointing back to the original \File{lib/}
  directory, or
  \item Set environment variable \Env{LD\_LIBRARY\_PATH} to point to the original \File{lib/} and \File{lib/condor/}
  directories
\end{itemize}
One of the new shared libraries, \File{libcondor\_utils\_7\_8\_0}, has no \File{.so}
versioning. Instead, the HTCondor version is included in the library name.
This means that an HTCondor binary must always be matched with the
\File{libcondor\_utils} library from the same HTCondor release.


\item  The \Condor{hdfs} service is no longer included within the HTCondor
	release.  Instead, the HTCondor + HDFS integration previously bundled with
	version 7.6 is available in version 7.8 as a \Term{Contribution Module}.
	Contribution Modules are optional packages that add functionality to
	HTCondor, but are provided and maintained outside of the core code base.  See
	the HTCondor Wiki at
	\URL{https://condor-wiki.cs.wisc.edu/index.cgi/wiki?p=ContribModules}.

\item Previous to version 7.8, by default the \Condor{master} would restart any
	individual daemon under its control if it notices that the file
	modification time of the binary for that daemon has changed.  Now the
	\Condor{master} will only monitor the file modification time of the
	\Condor{master} binary itself.  See section~\ref{sec:Pool-Upgrade}.  Also,
	see \MacroNI{MASTER\_NEW\_BINARY\_RESTART} on
	page~\pageref{param:MasterNewBinaryRestart}.

\item In DAGMan, if you have a PRE and a POST script on a node, the default now
	is that the POST script is run even if the PRE script failed.   This change
	could impact unaware workflows such that POST scripts might erroneously
	report the node as succeeded. You can get the old behavior by setting
	\MacroNI{DAGMAN\_ALWAYS\_RUN\_POST} to False.  In addition, you can no
	longer directly submit a rescue DAG file with \Condor{submit\_dag} unless
	\MacroNI{DAGMAN\_WRITE\_PARTIAL\_RESCUE} is set to False (not normally
	recommended).  See section~\ref{sec:DAGMan}.

\item The \MacroNI{KILL} expression cannot be used to grant more time to a job
	than offered by \Macro{MachineMaxVacateTime}. In HTCondor v7.8 and above, it
	is anticipated that most sites will simply use a default value of False for
	\MacroNI{KILL} and set \MacroNI{MachineMaxVacateTime} to control how long
	to wait.  See page~\pageref{param:MachineMaxVacateTime} for more
	information.


\end{itemize}


%%%      PLEASE RUN A SPELL CHECKER BEFORE COMMITTING YOUR CHANGES!
%%%      PLEASE RUN A SPELL CHECKER BEFORE COMMITTING YOUR CHANGES!
%%%      PLEASE RUN A SPELL CHECKER BEFORE COMMITTING YOUR CHANGES!
%%%      PLEASE RUN A SPELL CHECKER BEFORE COMMITTING YOUR CHANGES!
%%%      PLEASE RUN A SPELL CHECKER BEFORE COMMITTING YOUR CHANGES!

%%%%%%%%%%%%%%%%%%%%%%%%%%%%%%%%%%%%%%%%%%%%%%%%%%%%%%%%%%%%%%%%%%%%%%
\section{\label{sec:History-7-8}Stable Release Series 7.8}
%%%%%%%%%%%%%%%%%%%%%%%%%%%%%%%%%%%%%%%%%%%%%%%%%%%%%%%%%%%%%%%%%%%%%%

This is a stable release series of HTCondor.
As usual, only bug fixes (and potentially, ports to new platforms)
will be provided in future 7.8.x releases.
New features will be added in the 7.9.x development series.

The details of each version are described below.

%%%%%%%%%%%%%%%%%%%%%%%%%%%%%%%%%%%%%%%%%%%%%%%%%%%%%%%%%%%%%%%%%%%%%%
\subsection*{\label{sec:New-7-8-7}Version 7.8.7}
%%%%%%%%%%%%%%%%%%%%%%%%%%%%%%%%%%%%%%%%%%%%%%%%%%%%%%%%%%%%%%%%%%%%%%

\noindent Release Notes:

\begin{itemize}

\item HTCondor version 7.8.7 released on December 18, 2012.

\end{itemize}

\noindent New Features:

\begin{itemize}

\item None.

\end{itemize}

\noindent Configuration Variable and ClassAd Attribute Additions and Changes:

\begin{itemize}

\item None.

\end{itemize}

\noindent Bugs Fixed:

\begin{itemize}

\item Fixed a bug wherein running the \Condor{suspend} command on a scheduler
universe job would cause the \Condor{schedd} to crash.
\Ticket{3259}

\item HTCondor now places EC2 jobs on hold when they fail to authenticate,
rather than leaving them and other jobs with the same authentication tokens
idle indefinitely.
\Ticket{2274}

\item For EC2 resource jobs within the grid universe,
HTCondor now only destroys the user's half of the SSH keypair
when sure that the instance has terminated.  This prevents transient
problems from rendering an instance permanently inaccessible.
\Ticket{3289}

\item For EC2 resource jobs within the grid universe,
HTCondor no longer generates an SSH keypair if the user did not
request one.
\Ticket{3061}

\item HTCondor no longer generates SSH keypair names that are incompatible
with OpenStack v4 (Essex).
\Ticket{3060}

\item Jobs that were submitted with \Condor{submit} \Opt{-spool}
and failed during submission were 
left indefinitely in the queue in the Hold state.  
This has been fixed, such that these jobs are removed from the queue.
In addition, the \Condor{schedd} daemon will periodically 
check for jobs that have
been in Hold state due to failed file transfer for at least twelve hours;
these jobs will be removed from the queue.
\Ticket{3200}

\item Fixed a problem where an \Prog{ssh\_to\_job} or an interactive job 
session would be terminated prematurely if the execute machine 
was configured to track process trees via a dedicated login.
That is, when configuration variable 
\Macro{DEDICATED\_EXECUTE\_ACCOUNT\_REGEXP} is being used.  
\Ticket{3232}

\item ClassAd functions \Procedure{int} and \Procedure{real} now ignore 
trailing characters within
a string argument that contains a valid number.
\Ticket{3102}

\item Contrary to the intended behavior, jobs run via \Prog{glexec} did not
get put on hold shortly before their proxy expired.
\Ticket{3283}

\item Starting in version 7.8.0, when using \Prog{glexec}, 
the job was put on hold shortly after the user's proxy was refreshed.
The incorrect, but stated hold reason was, \Expr{"Proxy about to expire."}
\Ticket{3280}

\item The configuration variable
\Macro{DELEGATE\_JOB\_GSI\_CREDENTIALS\_REFRESH} had no effect,
as it was implemented using the incorrect variable name of
\MacroNI{DELEGATE\_JOB\_GSI\_CREDENTIALS\_RENEWAL}.
The implementation now uses the correct name.
\Ticket{3282}

\item When using privilege separation, jobs would be put on hold after
they finished running if the working directory contained links to
files that were not globally readable.
\Ticket{2904}

\item The configuration variable \MacroNI{ENABLE\_ADDRESS\_REWRITING}
was not correctly applied to the \Condor{schedd} address when
claiming a slot.  
The incorrect behavior observed was always as though
\MacroNI{ENABLE\_ADDRESS\_REWRITING} was \Expr{False}.
This could result
in communication errors for jobs running from multi-homed submit machines.
\Ticket{3330}

\item When using \Condor{defrag} or \Condor{drain}, a rare sequence of
events could result in the \Condor{startd} exiting with the
following error message:

\begin{verbatim}
ERROR "match_info() called with unexpected state (Drained)"
\end{verbatim}
\Ticket{3331}

\item The \Condor{gridmanager} no longer crashes when a CREAM grid job
is submitted with an X.509 proxy that does not have VOMS attributes.
\Ticket{3356}

\item \Condor{chirp} fetch could only transfer text files on Windows. It would
truncate or corrupt binary files.
\Ticket{3355}

\item Fixed a bug in the \Condor{gridmanager} that prevented the use
of grid-type \SubmitCmd{batch} to submit jobs into an HTCondor pool.
The \Condor{gridmanager} would attempt to use the wrong GAHP server.
\Ticket{3364}

\item The default for the undocumented configuration variable
\Macro{X\_RUNS\_HERE} was inverted from \Expr{True} to \Expr{False}
starting with the release of version 7.7.3. 
Its default has been reset to \Expr{True}. 
When \Expr{False}, the \Condor{master} will not start the \Condor{kbdd}.
\Ticket{3343}

\item Fixed a bug in the init script for Red Hat derived Linux systems that
prevented the Condor service from being stopped during system shutdown.
\Ticket{3368}

\item The \Condor{master} would sometimes crash on reconfiguration when the
High Availability configuration had changed. It no longer crashes.
\Ticket{3292}

\item Fixed a bug in the non-standard universe \Condor{shadow} that prevented parallel universe jobs from running when \verb|USE_NFS=True=| or \verb|USE_AFS=True=|. \Ticket{3361}

\end{itemize}

\noindent Known Bugs:

\begin{itemize}

\item None.

\end{itemize}

\noindent Additions and Changes to the Manual:

\begin{itemize}

\item The manual incorrectly identified configuration variable
\Macro{COLLECTOR\_PERSISTENT\_AD\_LOG} as \MacroNI{PERSISTENT\_AD\_LOG}.
This has now been corrected throughout the manual.
\Ticket{3205}

\end{itemize}


%%%%%%%%%%%%%%%%%%%%%%%%%%%%%%%%%%%%%%%%%%%%%%%%%%%%%%%%%%%%%%%%%%%%%%
\subsection*{\label{sec:New-7-8-6}Version 7.8.6}
%%%%%%%%%%%%%%%%%%%%%%%%%%%%%%%%%%%%%%%%%%%%%%%%%%%%%%%%%%%%%%%%%%%%%%

\noindent Release Notes:

\begin{itemize}

\item Condor version 7.8.6 released on October 25, 2012.

\item \Security This version contains an important security bug fix.  See below
for details of this and other bugs fixed.

\end{itemize}

\noindent Bugs Fixed:

\begin{itemize}

\item \Security Fixed a bug which allowed jobs submitted to the standard
universe to escalate privilege on the submit machine and execute code as root.
(CVE-2012-5390)

\end{itemize}


%%%%%%%%%%%%%%%%%%%%%%%%%%%%%%%%%%%%%%%%%%%%%%%%%%%%%%%%%%%%%%%%%%%%%%
\subsection*{\label{sec:New-7-8-5}Version 7.8.5}
%%%%%%%%%%%%%%%%%%%%%%%%%%%%%%%%%%%%%%%%%%%%%%%%%%%%%%%%%%%%%%%%%%%%%%

\noindent Release Notes:

\begin{itemize}

\item Condor version 7.8.5 released on October 22, 2012.

\end{itemize}


\noindent New Features:

\begin{itemize}

\item Condor now contains a tool called \Prog{accountant\_log\_fixer}, 
that can fix the damage to the file \File{Accountantnew.log} 
caused by a bug in the Condor version 7.8.4 \Condor{negotiator}.
\Ticket{3221}

\end{itemize}

\noindent Configuration Variable and ClassAd Attribute Additions and Changes:

\begin{itemize}

\item None.

\end{itemize}

\noindent Bugs Fixed:

\begin{itemize}

\item Fixed a problem with jobs that are submitted with
\begin{verbatim}
  noop_job = true
\end{verbatim}
These jobs would, in rare cases, cause the \Condor{schedd} daemon to crash.  
\Ticket{3156}

\item The \Condor{startd} daemon crashed if a job failed to match 
a partitionable slot after the application 
of configuration variables of the \MacroNI{MODIFY\_REQUEST\_EXPR\_} category.
\Ticket{3260}

\item The \Condor{schedd} daemon would mark scheduler universe jobs
as completed and remove them from the job queue,
even when they should have been requeued, according to policy.
This caused \Condor{dagman} jobs to fail to restart. This bug exists
in all Condor versions 7.8.0 through 7.8.4.
Upon upgrading from these Condor versions, users will need to intervene
in order to restart their \Condor{dagman} jobs. \Condor{dagman} should not
need such intervention when upgrading from Condor version 7.8.5.  
To restart a \Condor{dagman} job, 
the simplest solution is to issue the command
\begin{verbatim}
  condor_submit <DAGFile>.condor.sub
\end{verbatim}
where the original DAG was submitted with 
\begin{verbatim}
  condor_submit <DAGFile>
\end{verbatim}
\Ticket{3207}

\item The Condor version 7.8.4 \Condor{negotiator} daemon wrote 
corrupt resource entries to the file \File{Accountantnew.log},
which it would then not be able to read.
The Condor version 7.9.0 \Condor{negotiator} daemon will abort 
when trying to read these corrupted resource entries. 
The \Condor{negotiator} will now correct these corrupt
resource entries over time.
\Ticket{3221}

\item Fixed a bug in which the \Condor{schedd} statistics ClassAd attributes
\Attr{JobsAccumExecuteTime}
and \Attr{JobsAccumPostExecuteTime} were sometimes much too large for jobs
that had been vacated and then restarted.
Note that these currently undocumented attributes would only appear
in the ClassAd if the verbosity level for \Condor{schedd} statistics
was set at the high value of 2 by the configuration variable
\MacroNI{STATISTICS\_TO\_PUBLISH}.
\Ticket{3227}

\end{itemize}

\noindent Known Bugs:

\begin{itemize}

\item None.

\end{itemize}

\noindent Additions and Changes to the Manual:

\begin{itemize}

\item None.

\end{itemize}


%%%%%%%%%%%%%%%%%%%%%%%%%%%%%%%%%%%%%%%%%%%%%%%%%%%%%%%%%%%%%%%%%%%%%%
\subsection*{\label{sec:New-7-8-4}Version 7.8.4}
%%%%%%%%%%%%%%%%%%%%%%%%%%%%%%%%%%%%%%%%%%%%%%%%%%%%%%%%%%%%%%%%%%%%%%

\noindent Release Notes:

\begin{itemize}

\item Condor version 7.8.4 released on September 19, 2012.

\item This release contains several important security fixes and all users should upgrade as soon as possible.

\end{itemize}


\noindent New Features:

\begin{itemize}

\item None.

\end{itemize}

\noindent Configuration Variable and ClassAd Attribute Additions and Changes:

\begin{itemize}

\item The new configuration variable \Macro{GSI\_AUTHZ\_CONF}
fixes a bug in which an instance of Condor may utilize the  
wrong Globus mapping.
The configuration variable defines a path and file name 
to the file that contains the Globus mapping library. 
See the complete definition at
~\ref{param:GSIAuthzConf}.
\Ticket{2103}


\end{itemize}

\noindent Bugs Fixed:

\begin{itemize}

\item \Security Some code that was no longer used was removed.  The presence
of this code could expose information which would allow an attacker to control
another user's job.  (CVE-2012-3493)

\item \Security Some code that was no longer used was removed.  The presence
of this code could have lead to a Denial-of-Service attack which would allow
an attacker to remove another user's idle job.  (CVE-2012-3491)

\item \Security Filesystem (FS) authentication was improved to check the UNIX
permissions of the directory used for authentication.  Without this, an
attacker may have been able to impersonate another submitter on the same submit
machine.  (CVE-2012-3492)

\item \Security Although not user-visible, there were multiple updates to
remove places in the code where potential buffer overruns could occur, thus
removing potential attacks.  None were known to be exploitable.

\item \Security Although not user-visible, there were updates to the code to
improve error checking of system calls, removing some potential security
threats.  None were known to be exploitable.


% https://access.redhat.com/security/cve/CVE-2012-349X


\item Fixed the \Condor{schedd} daemon; 
it would crash when a submit description file
contained a malformed \verb@$$()@ expansion macro that contained
a period.
\Ticket{3216}

\item Fixed a case in which a daemon could crash and leave behind a log
file owned by \Login{root}.  This \Login{root}-owned file would then cause
subsequent attempts to restart the daemon to fail.
\Ticket{2894}

\item Fixed a special case bug in which configuration variables
defined utilizing initial substrings of \Expr{\$(DOLLAR)},
for example  \Expr{\$(D)} and  \Expr{\$(DO)},  
were not expanded properly.
\Ticket{3217}

\item The command \Condor{q} \Opt{-run} now displays correct HOST field 
information for local universe jobs.
\Ticket{3150}

\end{itemize}

\noindent Known Bugs:

\begin{itemize}

\item None.

\end{itemize}

\noindent Additions and Changes to the Manual:

\begin{itemize}

\item None.

\end{itemize}


%%%%%%%%%%%%%%%%%%%%%%%%%%%%%%%%%%%%%%%%%%%%%%%%%%%%%%%%%%%%%%%%%%%%%%
\subsection*{\label{sec:New-7-8-3}Version 7.8.3}
%%%%%%%%%%%%%%%%%%%%%%%%%%%%%%%%%%%%%%%%%%%%%%%%%%%%%%%%%%%%%%%%%%%%%%

\noindent Release Notes:

\begin{itemize}

\item Condor version 7.8.3 released on September 6, 2012.

\end{itemize}


\noindent New Features:

\begin{itemize}

\item The \File{libcondorapi} library for reading and writing job event
logs is again available as a shared library on Linux and Mac OS platforms.
Since Condor 7.5.x, it had only been available as a static library.
\Ticket{3047}

\end{itemize}

\noindent Configuration Variable and ClassAd Attribute Additions and Changes:

\begin{itemize}

\item To avoid the output of an unnecessary DAGMan error message,
the value of \Macro{DAGMAN\_LOG\_ON\_NFS\_IS\_ERROR}
is ignored when both \MacroNI{CREATE\_LOCKS\_ON\_LOCAL\_DISK}
and \MacroNI{ENABLE\_USERLOG\_LOCKING} are \Expr{True}.
\Ticket{3087}

\end{itemize}

\noindent Bugs Fixed:

\begin{itemize}

\item Fixed a bug in which usage of cgroups incorrectly included the
page cache in the maximum memory usage.
This bug fix is also included in Condor version 7.9.0.
\Ticket{3003}

\item Jobs from a hook to fetch work, 
where the hook is defined by configuration variable 
\MacroNI{<Keyword>\_HOOK\_FETCH\_WORK},
now correctly receive dynamic slots from a partitionable slot 
instead of claiming the entire partitionable slot.
\Ticket{2819}

\item Fixed a bug in which a slot might become stuck in the Preempting state
when a \Condor{startd} is configured with a hook to fetch work,
as defined by \Macro{<Keyword>\_HOOK\_FETCH\_WORK}.
\Ticket{3076}

\item Fixed a bug that caused Condor to transfer a job's input files from
the execute machine back to the submit machine as if they were output files.
This would happen if the
job's input files were stored in Condor's spool directory;
occurred if the job was submitted via Condor-C or via 
\Condor{submit} with the \Opt{-spool} or \Opt{-remote} options.
\Ticket{2406}

\item Fixed a bug that could cause the first grid-type cream jobs destined 
for a particular CREAM server to never be submitted to that server.
This bug was probably introduced in Condor version 7.6.5.
\Ticket{3054}

\item Fixed several problems with the XML parsing class
\Code{ClassAdXMLParser} in the ClassAds library:
  \begin{itemize}
  \item Several methods named \Procedure{ParseClassAd} were declared, 
  but never implemented. 
\Ticket{3049}
  \item The parser silently dropped leading white space in string values.
\Ticket{3042}
  \item The parser could go into an infinite loop or leak memory when
    reading a malformed ClassAd XML document. 
\Ticket{3045}
  \end{itemize}

\item Fixed a bug that prevented the \Opt{-f} command line option to
\Condor{history} from being recognized.
The \Opt{-f} option was being interpreted as \Opt{-forward}. 
At least four letters are now required for the \Opt{-forward} option
(\Opt{-forw}) to prevent ambiguity.
\Ticket{3044}

\item The implementation of the \Condor{history} \Opt{-backwards} option, 
which is the default ordering for reading the history file,
in the 7.7 series did not work on Windows platforms.
This has been fixed.
\Ticket{3055}

\item Fixed a bug that caused an invalid proxy to be delegated when
refreshing the job's X.509 proxy when configuration variable
\Macro{DELEGATE\_JOB\_GSI\_CREDENTIALS\_LIFETIME} was set to 0.
\Ticket{3059}

\item Fixed a bug in which DAGMan did not account properly for jobs being
suspended and then unsuspended.
\Ticket{3108}

\item \Condor{dagman} now takes note of job reconnect failed 
events (event code 24) in the user log, for counting idle jobs.
\Ticket{3189}

\item Job IDs generated by NorduGrid ARC 12.05 and above are now
properly recognized.
\Ticket{3062}

\item Fixed a bug in which Condor would not mark grid-type nordugrid jobs
as Running due to variation in the format of the job status value.
NorduGrid ARC job statuses of the form \Expr{INLRMS: ?} are now
properly recognized both with and without the space after the colon.
\Ticket{3118}

\item The \Condor{gridmanager} now properly handles X.509 proxy files
that are specified in the job ClassAd with a relative path name.
\Ticket{3027}

\item Fixed a bug that caused daemon names,
as set in configuration variables such as \MacroNI{STARTD\_NAME},
containing a period character to be ignored.
\Ticket{3172}

\item Fixed a bug that prevented the \Condor{schedd} from removing old
execute directories for local universe jobs on start up.
\Ticket{3176}

\item The \Condor{defrag} daemon sometimes scheduled fewer draining attempts 
than specified.
\Ticket{3199}

\item Fixed a bug that could cause the \Condor{gridmanager} to crash if a
grid universe job's X.509 user certificate did not contain an e-mail
address.
\Ticket{3203}

\item Fixed a bug introduced in Condor version 7.7.5 that caused multiple
\Condor{schedd} daemons running on the same machine to share the job queue
with each other due to way in which the default value of configuration
variable \MacroNI{JOB\_QUEUE\_LOG} was set.
\Ticket{3196}

\item Fixed a bug that could cause \Condor{q} to not print all jobs when
it thought it was querying an old \Condor{schedd} daemon.
\Ticket{3206}

\item Fixed a bug that could cause a job's standard output and standard
error files to be written in the job's initial working directory,
despite the submit description file's specification to write them 
to a different directory.
This would happen when the file transfer mechanism was used, 
the execution machine was running Condor version 7.7.1 or earlier, 
and either Condor's security negotiation
was disabled or the configuration variable
\MacroNI{SEC\_ENABLE\_MATCH\_PASSWORD\_AUTHENTICATION} was set to \Expr{True}.
\Ticket{3208}

\item The log message generated when the \MacroNI{EXECUTE} directory
is missing is now more helpful.
\Ticket{3194}

\item The load average was incorrect for non-English versions on 
Windows platforms.
This has been fixed for Windows Vista and more recent versions.
\Ticket{3182}

\end{itemize}

\noindent Known Bugs:

\begin{itemize}

\item None.

\end{itemize}

\noindent Additions and Changes to the Manual:

\begin{itemize}

\item There is now documentation for the submit description file commands
\SubmitCmd{encrypt\_input\_files},
\SubmitCmd{encrypt\_output\_files},
\SubmitCmd{dont\_encrypt\_input\_files}, and
\SubmitCmd{dont\_encrypt\_output\_files} in the \Condor{submit}
manual page.
These commands have been available since Condor version 6.7.2,
but were never documented.
See descriptions starting at
~\ref{man-condor-submit-dont-encrypt-input-files}.
\Ticket{3174}


\end{itemize}


%%%%%%%%%%%%%%%%%%%%%%%%%%%%%%%%%%%%%%%%%%%%%%%%%%%%%%%%%%%%%%%%%%%%%%
\subsection*{\label{sec:New-7-8-2}Version 7.8.2}
%%%%%%%%%%%%%%%%%%%%%%%%%%%%%%%%%%%%%%%%%%%%%%%%%%%%%%%%%%%%%%%%%%%%%%

\noindent Release Notes:

\begin{itemize}

\item Condor version 7.8.2 released on August 14, 2012.

\item \Security Fixed a critical problem with DNS handling.

\end{itemize}

\noindent New Features:

\begin{itemize}

\item None.

\end{itemize}

\noindent Configuration Variable and ClassAd Attribute Additions and Changes:

\begin{itemize}

\item None.

\end{itemize}

\noindent Bugs Fixed:

\begin{itemize}

\item \Security Fixed a critical problem with DNS handling.

\end{itemize}

\noindent Known Bugs:

\begin{itemize}

\item None.

\end{itemize}

\noindent Additions and Changes to the Manual:

\begin{itemize}

\item None.

\end{itemize}

%%%%%%%%%%%%%%%%%%%%%%%%%%%%%%%%%%%%%%%%%%%%%%%%%%%%%%%%%%%%%%%%%%%%%%
\subsection*{\label{sec:New-7-8-1}Version 7.8.1}
%%%%%%%%%%%%%%%%%%%%%%%%%%%%%%%%%%%%%%%%%%%%%%%%%%%%%%%%%%%%%%%%%%%%%%

\noindent Release Notes:

\begin{itemize}

\item Condor version 7.8.1 released on June 15, 2012.

\end{itemize}


\noindent New Features:

\begin{itemize}

\item None.

\end{itemize}

\noindent Configuration Variable and ClassAd Attribute Additions and Changes:

\begin{itemize}

\item (Added in 7.8.0.) The new configuration variable
\Macro{ENABLE\_DEPRECATION\_WARNINGS} causes \Condor{submit} to issue
warnings when a job requests features that are no longer supported.
\Ticket{2968}

\item (Added in 7.7.6) The new configuration variable
\Macro{BATCH\_GAHP} should be used instead of \Macro{PBS\_GAHP},
\Macro{LSF\_GAHP} and \Macro{SGE\_GAHP}. These older configuration
variables are still recognized, but their use is now discouraged.
\Ticket{2670}

\item The default value for \Macro{GROUP\_SORT\_EXPR} was changed 
so that the \Expr{<none>} group would always negotiate last 
when using hierarchical group quotas.
Associated with that, 
the default value for \Macro{NEGOTIATOR\_ALLOW\_QUOTA\_OVERSUBSCRIPTION} 
was changed to \Expr{True}.  
These changes were made to make negotiation behave more like it did 
in the stable 7.4 series of Condor,
before hierarchical group quotas were added.
\Ticket{3040}

\end{itemize}

\noindent Bugs Fixed:

\begin{itemize}

\item Fixed a bug that caused events to not be written to the job event
log when the log is written in XML and a job policy expression triggering
the event contains any double quote marks.
\Ticket{3048}

\item Fixed a bug in the Condor init script that would cause
the init script to hang if Condor was not running.
\Ticket{2872}

\item Fixed a bug that caused parallel universe jobs using
Parallel Scheduling Groups 
(see section ~\ref{sec:Configure-Dedicated-Groups})
to occasionally stay idle even when
there were available machines to run them.
\Ticket{3017}

\item Fixed a bug that caused the \Condor{gridmanager} to crash when
attempting to submit jobs to a local PBS, LSF, or SGI cluster.
\Ticket{3014}

\item Fixed a bug in the handling of local universe jobs which caused
the \Condor{schedd} to log a spurious \Expr{ERROR} message
every time a local universe job exited, 
and then further caused the statistics for local universe jobs to be 
incorrectly computed.
\Ticket{3008}

\item Changed the internally used \Condor{ckpt\_probe} executable
to link statically, which should make the
checkpoint signature more resistant to non-significant changes in the system
configuration.
\Ticket{2901}

\item Restored Globus and VOMS support for the Mac OS X platform.
\Ticket{2991}

\item Fixed a bug when Condor runs under the PrivSep model,
in which if a job created a hard link from one file to another,
Condor was unable to transfer the files back to the submit side,
and the job was put on hold.
\Ticket{2987}

\item When configuration variables \MacroNI{MaxJobRetirementTime} or
\MacroNI{MachineMaxVacateTime} were very large, estimates of machine
draining badput and completion time were sometimes nonsensical
because of integer overflow.
\Ticket{3001}

\item Fixed a bug where per-job subdirectories and their contents
in \MacroUNI{SPOOL} would not be removed when the associated job
left the queue.
\Ticket{2942}

\item Fixed a bug that could cause the \Condor{schedd} to 
occasionally crash due to a race condition when running local universe jobs.
Associated with the bug would be the error message
\footnotesize
\begin{verbatim}
No local universe jobs were expected to be running, but one just exited!
\end{verbatim}
\normalsize
\Ticket{3009}

\end{itemize}

\noindent Known Bugs:

\begin{itemize}

\item None.

\end{itemize}

\noindent Additions and Changes to the Manual:

\begin{itemize}

\item Submit description file commands introduced in Condor version 7.7.1
have now been documented.
See the \Condor{submit} manual page at ~\ref{man-condor-submit} for
the newly added definitions of
\begin{description}
  \item[\SubmitCmd{ec2\_availability\_zone}]
  \item[\SubmitCmd{ec2\_ebs\_volumes}]
  \item[\SubmitCmd{ec2\_elastic\_ip}]
  \item[\SubmitCmd{ec2\_keypair\_file}]
  \item[\SubmitCmd{ec2\_vpc\_ip}]
  \item[\SubmitCmd{ec2\_vpc\_subnet}]
\end{description}

\item There is now a manual page for \Condor{router\_rm}, 
a script that provides additional features convenient for removing
jobs managed by the Condor Job Router.

\item Documentation not completed for the 7.7.6 release is now available.
The use of configuration variable \MacroNI{BATCH\_GAHP},
as well as the use of the new \SubmitCmd{grid\_resource} of
type \Expr{batch} for local submission of PBS, LSF, and SGE
jobs is documented.
See section ~\ref{sec:batch} for details.
\Ticket{2670}

\end{itemize}


%%%%%%%%%%%%%%%%%%%%%%%%%%%%%%%%%%%%%%%%%%%%%%%%%%%%%%%%%%%%%%%%%%%%%%
\subsection*{\label{sec:New-7-8-0}Version 7.8.0}
%%%%%%%%%%%%%%%%%%%%%%%%%%%%%%%%%%%%%%%%%%%%%%%%%%%%%%%%%%%%%%%%%%%%%%

\noindent Release Notes:

\begin{itemize}

\item Condor version 7.8.0 released on May 10, 2012.

\end{itemize}


\noindent New Features:

\begin{itemize}

\item (Added in 7.7.6.)  The new \Arg{-\_condor\_relocatable} argument
may be given as part of the invocation of a program that uses
standalone checkpointing.  This allows checkpointed programs to restart
without attempting to change to their original directory.
\Ticket{2877}

\item (Added in 7.7.5.) Added the \Arg{-absent} flag to \Condor{status},
which displays absent ClassAds.
\Ticket{2690}

\item (Added in 7.7.5.) Implement absent ads, which help track pool membership
in a persistent way.
\Ticket{2608} 

\end{itemize}

\noindent Configuration Variable and ClassAd Attribute Additions and Changes:

\begin{itemize}

\item The job ClassAd attribute \Attr{RemotePool} is now saved in
  \Attr{LastRemotePool} when the job finishes running.

\end{itemize}

\noindent Bugs Fixed:

\begin{itemize}

\item (Fixed in 7.7.6.) Fix \Arg{-absent}, \Arg{-vm}, and \Arg{-java}
flags to \Condor{status} so that they work with the \Arg{-long} option.
\Ticket{2943}

\item Support glob() on Scientific Linux 6 and others using the new
Linux system call fstatat(), but only when not using remote system calls.
\Ticket{2945}

\item Fixed potential startd crash introduced in v7.7.5 when claiming 
a partitionable slot that was in the Owner state. 
\Ticket{2936}

\item When ClassAd function stringListMember() is called with an empty
string as the second argument, it now evaluates to \Expr{False}.
Previously, it incorrectly evaluated to \Expr{Undefined}.
\Ticket{2953}

\item Format tags \%v and \%V for the \Opt{-format} option now properly
print all ClassAd value types. Previously, \Expr{True} and \Expr{False}
were printed as integers, and new ClassAd types like lists and nested
ClassAds could not be printed.
\Ticket{2960}

\item Restored RCS keyword strings CondorVersion and CondorPlatform to
the Condor binaries. These strings are found and printed by the 
\Opt{ident} program on Unix. They were missing in Condor versions 7.7.3
and later.
\Ticket{2932}

\item \Condor{job\_router} failed to route spooled source jobs.
\Ticket{2955}

\item Fixed a bug on Debian 6 and RHEL 6 that could cause standard
universe jobs to never checkpoint. This would happen if the job
triggered a call to NSCD (Name Service Caching Daemon) but NSCD 
wasn't running. 
Calls to NSCD can be triggered by a look up of a user account or
resolving a machine hostname to an IP address.
Now, NSCD is never consulted by a standard universe
job (this was already the behavior on other platforms).
\Ticket{2973}

\end{itemize}

\noindent Known Bugs:

\begin{itemize}

\item None.

\end{itemize}

\noindent Additions and Changes to the Manual:

\begin{itemize}

\item None.

\end{itemize}



\input{version-history/7-7.history.tex}
\input{version-history/7-6.history.tex}
% as of April 2012, Karen no longer wants to include these older
% version histories with the 7.4 and 7.5 manuals.
%\input{version-history/7-5.history.tex}
%\input{version-history/7-4.history.tex}
% as of April 2011, Karen no longer wants to include these older
% version histories with the 7.6 and beyond manuals.
%\input{version-history/7-3.history.tex}
%\input{version-history/7-2.history.tex}
%\input{version-history/7-1.history.tex}
%\input{version-history/7-0.history.tex}
% Oct 2009, as we release 7.4, Karen commented out inclusion of the
% 6.9 and 6.8 histories
%\input{version-history/6-9.history.tex}
%\input{version-history/6-8.history.tex}
% Dec 2007, as we release 7.x, Karen commented out the older histories
%\input{version-history/6-7.history.tex}
%\input{version-history/6-6.history.tex}
% Feb 2007 -- still in the manual source, just not incorporating
% these old histories into the finished product, thereby
% reducing the size of the manual by 200 pages
%\input{version-history/6-5.history.tex}
%\input{version-history/6-4.history.tex}
%\input{version-history/6-3.history.tex}
%\input{version-history/6-2.history.tex}
%\input{version-history/6-1.history.tex}
%\input{version-history/6-0.history.tex}
