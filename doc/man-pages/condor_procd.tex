\begin{ManPage}{\label{man-condor-procd}\Condor{procd}}{1}
{Track and manage process families}
\Synopsis \SynProg{\Condor{procd}}
\Opt{-h}

\SynProg{\Condor{procd}}
\OptArg{-A}{address-file}
\oOpt{options}

\index{HTCondor commands!condor\_procd}
\index{condor\_procd command}

\Description 

\Condor{procd} tracks and manages process families on behalf of the
HTCondor daemons. 
It may track families of PIDs via relationships such
as: direct parent/child, environment variables, UID, and supplementary
group IDs.  
Management of the PID families include 
\begin{itemize}
\item registering new families or new members of existing families
\item getting usage information
\item signaling families for operations such as suspension, 
continuing, or killing the family 
\item getting a snapshot of the tree of families 
\end{itemize}
In a regular HTCondor installation, 
this program is not intended to be used or executed by any human.

The required argument, 
\OptArg{-A}{address-file},
is the path and file name of the address file which is the named pipe
that clients must use to speak with the \Condor{procd}.
	
\begin{Options}

  \OptItem{\Opt{-h}}{Print out usage information and exit.}
	
  \OptItem{\Opt{-D}}{Wait for the debugger. Initially sleep 30
  seconds before beginning normal function.}

  \OptItem{\OptArg{-C}{principal}}{The \Arg{principal} is the UID of the owner
  of the named pipe that clients must use to speak to the \Condor{procd}.}
	
  \OptItem{\OptArg{-L}{log-file}}{A file the \Condor{procd} will
  use to write logging information.}
	
  \OptItem{\Opt{-E}}{ When specified,
  another tool such as the \Prog{procd\_ctl} tool must allocate the GID
  associated with a process.
  When this option is \emph{not} specified, 
  the \Condor{procd} will allocate the GID itself.  }

  \OptItem{\OptArg{-P}{PID}}{If not specified, the \Condor{procd} will 
  use the \Condor{procd}'s parent, which may not be PID 1 on Unix,
  as the parent of the \Condor{procd} and the root of the tracking
  family. When not specified, if the \Condor{procd}'s parent PID
  dies, the \Condor{procd} exits. When specified, the \Condor{procd} will
  track this \Arg{PID} family in question and not also exit if
  the PID exits.}

  \OptItem{\OptArg{-S}{seconds}}{The maximum number of seconds the 
  \Condor{procd} will wait between taking snapshots
  of the tree of families. Different clients to the
  \Condor{procd} can specify different snapshot times. The
  quickest snapshot time is the one performed by the
  \Condor{procd}. When this option is not specified, a default value
  of 60 seconds is used.}

  \OptItem{\OptArg{-G}{min-gid max-gid}}{If the \Opt{-E} option
  is \emph{not} specified, 
  then track process families using a self-allocated, free GID out of the
  inclusive range specified by \Arg{min-gid} and \Arg{max-gid}.
  This means that if a new process shows up using a previously known GID,
  the new process will automatically associate into the
  process family assigned that GID. 
  If the \Opt{-E} option \emph{is} specified,
  then instead of self-allocating the GID,
  the \Prog{procd\_ctl} tool must be
  used to associate the GID with the PID root of the family. 
  The associated GID must still be in the range specified. 
  This is a Linux-only feature.}
		
  \OptItem{\OptArg{-K}{windows-softkill-binary}}{This is the path and
  executable name of the \Prog{condor\_softkill.exe} binary.
  It is used to send softkill signals to process families. 
  This is a Windows-only feature.}

  \OptItem{\OptArg{-I}{glexec-kill-path glexec-path}}{ Specifies,
  with \Arg{glexec-kill-path}, the path and executable name of a
  binary used to send a signal to a PID. 
  The \Prog{glexec} binary,
  specified by \Arg{glexec-path}, 
  executes the program specified with \Arg{glexec-kill-path} 
  under the right privileges to send the signal.}
	
\end{Options}

\GenRem

This program may be used in a stand alone mode, independent of
HTCondor, to track process families. The programs \Prog{procd\_ctl} and
\Prog{gidd\_alloc} are used with the \Condor{procd} in stand alone mode
to interact with the daemon and to inquire about certain state of running
processes on the machine, respectively.

\ExitStatus

\Condor{procd} will exit with a status value of 0 (zero) upon success,
and it will exit with the value 1 (one) upon failure.

\end{ManPage}
