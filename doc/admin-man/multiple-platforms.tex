%%%%%%%%%%%%%%%%%%%%%%%%%%%%%%%%%%%%%%%%%%%%%%%%%%%%%%%%%%%%%%%%%%%%%%%%%%%
\subsection{\label{sec:Multiple-Platforms}Configuring HTCondor for
Multiple Platforms} 
%%%%%%%%%%%%%%%%%%%%%%%%%%%%%%%%%%%%%%%%%%%%%%%%%%%%%%%%%%%%%%%%%%%%%%%%%%

A single, global
configuration file may be used for all platforms in an HTCondor pool, with only
platform-specific settings placed in separate files.  This greatly
simplifies administration of a heterogeneous pool by allowing
changes of platform-independent, global settings in one place, instead of
separately for each platform.  This is made possible by treating the
\Macro{LOCAL\_CONFIG\_FILE} configuration variable as a
list of files, instead of a single file.  Of course, this only
helps when using a shared file system for the machines in the
pool, so that multiple machines can actually share a single set of
configuration files.

With multiple platforms, put all
platform-independent settings (the vast majority) into the regular
\File{condor\_config} file, which would be shared by all platforms.
This global file would be the one that is found with the
\Env{CONDOR\_CONFIG} environment variable, the user \Login{condor}'s home
directory, or \File{/etc/condor/condor\_config}.
Then, set the \MacroNI{LOCAL\_CONFIG\_FILE} configuration variable from that
global configuration file to specify both a platform-specific
configuration file and
optionally, a local, machine-specific configuration file.

The order of file specification in the
\MacroNI{LOCAL\_CONFIG\_FILE} configuration variable is important,
because settings
in files at the beginning of the list are overridden if the same
settings occur in files later within the list.  So, if specifying the
platform-specific file and then the machine-specific file, settings in
the machine-specific file would override those in the
platform-specific file (as is likely desired).  

%%%%%%%%%%%%%%%%%%%%%%%%%%%%%%%%%%%%%%%%%%%%%%%%%%%%%%%%%%%%%%%%%%%%%%%%%%%
\subsubsection{\label{sec:Specify-Platform-Files}Utilizing a
Platform-Specific Configuration File} 
%%%%%%%%%%%%%%%%%%%%%%%%%%%%%%%%%%%%%%%%%%%%%%%%%%%%%%%%%%%%%%%%%%%%%%%%%%%

The name of 
platform-specific configuration files may be specified by using the
\MacroNI{ARCH} and \MacroNI{OPSYS} configuration variables, as are defined
automatically by HTCondor.
For example, for 32-bit Intel Windows 7
machines and 64-bit Intel Linux machines,
the files ought to be named:

\begin{verbatim}
  condor_config.INTEL.WINDOWS
  condor_config.X86_64.LINUX
\end{verbatim}

Then, assuming these files are in the directory defined by the
\MacroNI{ETC} configuration variable,
and machine-specific configuration files are in
the same directory, named by each machine's host name, the
\Macro{LOCAL\_CONFIG\_FILE} configuration macro should be:

\footnotesize
\begin{verbatim}
LOCAL_CONFIG_FILE = $(ETC)/condor_config.$(ARCH).$(OPSYS), \
                    $(ETC)/$(HOSTNAME).local
\end{verbatim}
\normalsize

Alternatively, when using AFS, an \Expr{@sys} link may be used to
specify the platform-specific configuration file,
which lets AFS resolve this link based on platform name.
For example, consider
a soft link named \File{condor\_config.platform} that points to
\File{condor\_config.@sys}.  In this case, the files might be named:

\begin{verbatim}
  condor_config.i386_linux2
  condor_config.platform -> condor_config.@sys
\end{verbatim}

and the \MacroNI{LOCAL\_CONFIG\_FILE} configuration variable would be set to

\footnotesize
\begin{verbatim}
LOCAL_CONFIG_FILE = $(ETC)/condor_config.platform, \
                    $(ETC)/$(HOSTNAME).local
\end{verbatim}
\normalsize

%%%%%%%%%%%%%%%%%%%%%%%%%%%%%%%%%%%%%%%%%%%%%%%%%%%%%%%%%%%%%%%%%%%%%%%%%%%
\subsubsection{\label{sec:Platform-Specific-Settings}Platform-Specific
Configuration File Settings}
%%%%%%%%%%%%%%%%%%%%%%%%%%%%%%%%%%%%%%%%%%%%%%%%%%%%%%%%%%%%%%%%%%%%%%%%%%%

The configuration variables that are truly platform-specific are:

\begin{description}

\item[\Macro{RELEASE\_DIR}] Full path to to the installed
  HTCondor binaries.  While the configuration files may be shared among
  different platforms, the binaries certainly cannot.  Therefore,
  maintain separate release directories for each platform
  in the pool.  

\item[\Macro{MAIL}] The full path to the mail program.  

\item[\Macro{CONSOLE\_DEVICES}] Which devices in \File{/dev} should be
  treated as console devices.

\item[\Macro{DAEMON\_LIST}] Which daemons the \Condor{master} should
  start up.  The reason this setting is platform-specific is
  to distinguish the \Condor{kbdd}.
  It is needed on many Linux and Windows machines,
  and it is not needed on other platforms.

\end{description}

Reasonable defaults for all of these configuration variables
will be found in the
default configuration files inside a given platform's binary distribution
(except the \MacroNI{RELEASE\_DIR}, since 
the location of the HTCondor binaries and libraries is installation specific).
With multiple platforms,
use one of the \File{condor\_config} files from
either running \Condor{configure} or from the
\File{\MacroUNI{RELEASE\_DIR}/etc/examples/condor\_config.generic} file,
take these settings out,
save them into a platform-specific file,
and install the resulting platform-independent file as the global
configuration file.
Then,
find the same settings from the configuration files for any other platforms
to be set up, and put them in their own platform-specific files.
Finally, set the \MacroNI{LOCAL\_CONFIG\_FILE} configuration variable
to point to
the appropriate platform-specific file, as described above.

Not even all of these configuration variables are necessarily
going to be different.
For example, if an installed mail program understands the
\Opt{-s} option in \File{/usr/local/bin/mail} on all platforms,
the \MacroNI{MAIL} macro may be set to that in the global configuration
file, and not define it anywhere else.
For a pool with only Linux or Windows machines,
the \MacroNI{DAEMON\_LIST} will be the same for each, so there is no
reason not to put that in the global configuration file.

%%%%%%%%%%%%%%%%%%%%%%%%%%%%%%%%%%%%%%%%%%%%%%%%%%%%%%%%%%%%%%%%%%%%%%%%%%%
\subsubsection{\label{sec:Other-Uses-for-Platform-Files}Other Uses for
Platform-Specific Configuration Files} 
%%%%%%%%%%%%%%%%%%%%%%%%%%%%%%%%%%%%%%%%%%%%%%%%%%%%%%%%%%%%%%%%%%%%%%%%%%%

It is certainly possible that an installation may want other 
configuration variables to be platform-specific as well.
Perhaps a different policy is desired for
one of the platforms.
Perhaps different people should get the
e-mail about problems with the different platforms.
There is nothing hard-coded about any of this.
What is shared and
what should not shared is entirely configurable.

Since the \Macro{LOCAL\_CONFIG\_FILE} macro can be an arbitrary
list of files, an installation can even break up the global,
platform-independent settings into separate files.
In fact, the global configuration file might
only contain a definition for \MacroNI{LOCAL\_CONFIG\_FILE}, and all
other configuration variables would be placed in separate files.  

Different people may be given different permissions to change different
HTCondor settings.  For example, if a user is to be able to
change certain settings, but nothing else, those
settings may be placed in a file which was early
in the \MacroNI{LOCAL\_CONFIG\_FILE} list,
to give that user write permission on that file.
Then, include all the other files after that one.
In this way, if the user was attempting to
change settings that the user should not be permitted to change,
the settings would be overridden.  

This mechanism is quite flexible and powerful.  For
very specific configuration needs, they can probably be met by
using file permissions, the \MacroNI{LOCAL\_CONFIG\_FILE} configuration
variable, and imagination.

